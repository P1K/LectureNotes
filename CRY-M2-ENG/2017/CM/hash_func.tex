\documentclass[xcolor=table,usenames,dvipsnames,compress]{beamer}
\usepackage[english]{babel}
\usepackage{verbatim}
\usepackage{multirow}
\usepackage{xspace}
\usepackage{textcomp}
\usepackage{bbold}
\usepackage{MnSymbol}
\usecolortheme{rose}
\usepackage{beamerthemeuclcryptov2}
\usepackage[utf8]{inputenc}
\usepackage{xcolor}
\usepackage{pgf}
\usepackage{rotating}
\usepackage{tikz}
\usepackage{listings}
\usepackage{relsize}
\usepackage{tabularx}
\usepackage{booktabs}
\usepackage{yfonts}
\usepackage{fourier-orns}
\usepackage[normalem]{ulem}

\setbeamercolor{section in head/foot}{fg=structure.fg!75}
\useinnertheme{default}
\useoutertheme[subsection=false]{miniframes}
%\usefonttheme{structureitalicserif}
\usefonttheme{default}
\setbeamertemplate{enumerate items}[square]
\beamertemplatetransparentcovered

% to remove head sections
\setbeamertemplate{headline}[text line]{\phantom{Bouh}}



%% Public TikZ libraries
\usetikzlibrary{positioning}
\usetikzlibrary{shapes}
\usetikzlibrary{patterns}

%% Custom TikZ addons
\usetikzlibrary{crypto.symbols}
\tikzset{shadows=no}        % Option: add shadows to XOR, ADD, etc.

\definecolor{llgray}{rgb}{0.9,0.9,0.9}
\definecolor{dgray}{rgb}{0.4,0.4,0.4}
\definecolor{lgreen}{rgb}{0.05,0.7,0.3}
\definecolor{lblue}{rgb}{0.3,0.7,0.9}
\definecolor{llgreen}{rgb}{0.77,0.95,0.77}
\definecolor{Lgreen}{rgb}{0.3,0.7,0.3}

%
\newcommand\ie{\emph{i.e.}\xspace}
\newcommand\eg{\emph{e.g.}\xspace}
\newcommand\etal{\emph{\& al.}\xspace}
\newcommand\defas{:=}
\newcommand\randraw{\xleftarrow{{\scriptscriptstyle\$}}}
%
\DeclareMathOperator\hash{\mathcal{H}}
\DeclareMathOperator\E{\mathcal{E}}
\DeclareMathOperator\cf{\textswab{f}}
\DeclareMathOperator\perm{\textswab{p}}
\DeclareMathOperator\bigo{\mathcal{O}}
%
\newcommand\ms{\mathcal{M}}
\newcommand\ds{\mathcal{D}}
\newcommand\iv{\text{IV}}
%
\AtBeginSection[] {
 \begin{frame}<beamer>{}
   \tableofcontents[currentsection,currentsubsection]
 \end{frame}
}

%\lstset{emph={x}, emphstyle=\color{Violet},
%emph={-}, emphstyle=\color{Green},
%}

\title[Hash functions]{Crypto Engineering (GBX9SY03)\\
\decosix\\
Hash functions}

\author[]{Pierre Karpman\\
\url{pierre.karpman@univ-grenoble-alpes.fr}\\
{\footnotesize\url{https://www-ljk.imag.fr/membres/Pierre.Karpman/tea.html}}}

\date{2017--10--18}

\begin{document}

\begin{frame}
  \titlepage
\end{frame}

% \begin{frame}
%   \tableofcontents
% \end{frame}
%
%\section{Introduction}

\begin{frame}{First definition}
\begin{exampleblock}{Hash function}
A hash function is a mapping $\hash : \ms \rightarrow \ds$
\end{exampleblock}
{\footnotesize So it really is just a function...}
\medskip

Usually:
\begin{itemize}
\item $\ms = \bigcup_{\ell < N} \{0,1\}^\ell$, $\ds = \{0,1\}^n$, $N \gg n$
\item $N$ is typically $\geq 2^{64}$, $n \in \{\text{\xout{128}, \xout{160}, \sout{224}, 256, 384, 512}\}$
\end{itemize}
{\footnotesize Also popular now: extendable-output functions (XOFs): $\ds = \bigcup_{\ell < N'} \{0,1\}^\ell$ }

\begin{itemize}
\item Hash functions are \emph{keyless}
\item So, how do you tell if one's \textbf{good}?
\end{itemize}
\end{frame}

\begin{frame}{Idealized hash functions: Random oracles}
\begin{exampleblock}{Random oracle}
A function $\hash : \ms \rightarrow \ds$ s.t. $\forall x \in \ms$, $\hash(x) \randraw \ds$
\end{exampleblock}
\begin{itemize}
\item ``The best we can ever get''
\item Sometimes useful in proofs (``Random oracle model'', or ROM)
\item Not possible to have one {\footnotesize(except for small (co-)domains assuming a TRNG)}
\item But we can get \emph{approximations} (e.g. SHA-3)
\end{itemize}
\end{frame}

\begin{frame}{Main security properties}
What is hard for a RO should be hard for any HF\\
$\Rightarrow$
\begin{enumerate}
\item \textbf{First preimage}: given $t$, find $m$ s.t. $\hash(m) = t$
\item \textbf{Second preimage}: given $m$, find $m' \neq m$ s.t. $\hash(m) = \hash(m')$
\item \textbf{Collision}: find $(m, m' \neq m)$ s.t. $\hash(m) = \hash(m')$
\end{enumerate}

\medskip

Generic complexity:\\
1), 2): $\Theta(2^n)$;\\
3): $\Theta(2^{n/2})$ $\leftlsquigarrow$ ``Birthday paradox''
\end{frame}

\begin{frame}{Why do we care? Applications!}
Hash functions are useful for:
\begin{itemize}
\item Hash-and-sign (RSA signatures, (EC)DSA, ...)
\item Message-authentication codes (HMAC, ...)
\item Password hashing (with a grain of salt)
\item Hash-based signatures (inefficient but PQ)
\item As ``RO instantiations'' (OAEP, ...)
\item As one-way functions (OWF)
\end{itemize}
\end{frame}
% TODO expand?

\begin{frame}{So, how do you build hash functions?}
\begin{itemize}
\item Objective \#1: be secure
\item Objective \#2: be efficient
\begin{itemize}
\item At most a few dozen cycles/byte!
\item $\Rightarrow$ work with limited amount of memory
\end{itemize}
\end{itemize}
So...
\begin{itemize}
\item (\#2) Build $\hash$ from a \emph{small component}
\item (\#1) Prove that this is okay
\end{itemize}
\end{frame}

\begin{frame}{What kind of small component?}
\begin{exampleblock}{Compression function}
A compression function is a mapping $\cf : \{0,1\}^n \times \{0,1\}^b \rightarrow \{0,1\}^n$
\end{exampleblock}
\begin{itemize}
\item A family of functions from $n$ to $n$ bits
\item Not unlike a block cipher, only not invertible
\end{itemize}
\begin{exampleblock}{Permutation}
A permutation is an invertible mapping $\perm : \{0,1\}^n \rightarrow \{0,1\}^n$
\end{exampleblock}
{\footnotesize Yes, very simple}
\begin{itemize}
\item Like a block cipher with a fixed key, \eg $\perm = \E(0,\cdot)$
\end{itemize}
\end{frame}

\begin{frame}{From small to big (compression function case)}
Assume a good $\cf$

\medskip

\begin{itemize}
\item Main problem: fixed-size domain $\{0,1\}^n \times \{0,1\}^b$
\item Objective: \emph{domain extension} to $\bigcup_{\ell < N} \{0,1\}^\ell$
\item (Not unlike using a mode of operation with a BC)
\end{itemize}

\medskip

The classical answer: the Merkle-Damg\aa rd construction (1989)
\end{frame}

\begin{frame}{MD: with a picture}
\begin{figure}
	\input{md.tex}
\end{figure}

That is: $\hash(m_1||m_2||m_3||\ldots) = \cf(\ldots\cf(\cf(\cf(\iv, m_1), m_2), m_3), \ldots)$

\medskip

pad($m$) $\approx$ $m||1000\ldots00\langle\text{length of }m\rangle$
\end{frame}

\begin{frame}{MD: does it work?}
Efficiency?
\begin{itemize}
\item Only sequential calls to $\cf$
\item $\Rightarrow$ fine
\end{itemize}

Security?
\begin{itemize}
\item Still to be shown
\item Objective: \emph{reduce} security of $\hash$ to that of $\cf$
\begin{itemize}
\item ``If $\cf$ is good, then $\hash$ is good''
\end{itemize}
\item True for collision and first preimage, \textbf{false} for second preimage
\end{itemize}
\end{frame}

\begin{frame}{MD (partial) security proof}
Method: simple contrapositive argument
\begin{itemize}
\item Attack \{$1^\text{st}$preim., coll.\} on $\hash \Rightarrow$ attack \{$1^\text{st}$preim., coll.\} on $\cf$
\end{itemize}

\begin{alertblock}{First preimage case}
If $\hash(m_1||m_2||\ldots||m_\ell) = t$, then $\cf(\hash(m_1||m_2||\ldots||m_{\ell-1}), m_\ell) = t$
\end{alertblock}

\begin{alertblock}{Collision case (sketch)}
If $\hash(m_1||m_2||\ldots||m_\ell) = \hash(m'_1||m'_2||\ldots||m'_\ell)$, show that $\exists i$ s.t.
$(h_i \defas \hash(m_1||m_2||\ldots||m_{i-1}), m_i) \neq (h'_i \defas \hash(m'_1||m'_2||\ldots||m'_{i-1}), m'_i)$ and
$\cf(h_i, m_i) = \cf(h'_i, m'_i)$
\end{alertblock}
\begin{itemize}
\item Proper message padding useful to make it work!
\end{itemize}
\end{frame}

\begin{frame}{What about $2^\text{nd}$ preimages??}
No proof (with optimal resistance), can't have one:
\begin{itemize}
\item Generic attack on messages of $2^k$ blocks for a cost $\approx k 2^{n/2+1} + 2^{n-k+1}$ (Kelsey and Schneier, 2005)
\item Idea: exploit internal collisions in the $h_i$s
\end{itemize}
This is not nice, but:
\begin{itemize}
\item Requires (very) long messages to gain something
\item At least as expensive as collision search
\item If $n$ is chosen s.t. generic collisions are out of reach, we're somewhat fine
\end{itemize}
$\Rightarrow$ Didn't make people give up MD hash functions (MD5, SHA-1, SHA-2 family)
\end{frame}

\begin{frame}{Attack with an expandable message}

\begin{center}
\includegraphics{2nd-preimage}
\end{center}

\bigskip

\url{https://www.iacr.org/authors/tikz/}
\end{frame}


\begin{frame}{Is that unavoidable?}
No! Simple patch: Chop-MD/Wide-pipe MD (Coron \etal, 2005) and (Lucks, 2005)
\begin{itemize}
\item Build $\hash$ from $\cf : \{0,1\}^{2n} \times \{0,1\}^b \rightarrow \{0,1\}^{2n}$, truncate output to $n$ bits (say)
\item Collision in the output $\nRightarrow$ collision in the internal state
\item Very strong provable guarantees (Coron \etal)
\begin{itemize}
\item Secure domain extender for fixed-size RO
\end{itemize}
\item Concrete instantiations: SHA-512/224, SHA-512/256 (2015)
\end{itemize}
\end{frame}

\begin{frame}{Practical impact of the MD proof}

\begin{itemize}
\item If one can't attack $\cf$ underlying $\hash$, all is well
\item Else, ...???
\item $\Rightarrow$ Attacking $\cf$ is a meaningful goal for cryptographers ($\approx$ (semi-)freestart attacks)
\item Ideally, \emph{never} use a $\hash$ with broken $\cf$
\end{itemize}

\end{frame}

\begin{frame}{The MD5 failure}
\vspace{-6mm}
\begin{itemize}
\item MD5: designed by Rivest (1992)
\item 1993: very efficient collision attack on the compression function (den Boer and Bosselaers); mean time of 4 minutes on a 33\,MHz 80386
\item MD5 still massively used...
\item 2005: very efficient collision attack on the \emph{hash function} (Wang and Yu)
\item Still used...
\item 2007: practically threatening collisions (Stevens \etal)
\item Still used...
\item 2009: even worse practical collision attacks (Stevens \etal)
\item Hmm, maybe we should move on?
\end{itemize}
\end{frame}

\begin{frame}{Was this avoidable?}
Yes!
\begin{itemize}
\item Early signs of weaknesses $\Rightarrow$ move to alernatives ASAP!
\item What were they (among others)?
\begin{itemize}
\item 1992: {\color{Red}RIPEMD} (RIPE); practically broken (collisions) 2005 (Wang \etal)
\item 1993: {\color{Red}SHA-0} (NSA); broken (collisions) 1998 (Chabaud and Joux); practically broken 2005 (Biham \etal)
\item 1995: {\color{RedOrange}SHA-1} (NSA); broken (collisions) 2005 (Wang \etal); practically broken 2017 (Stevens \etal (and me!))
\item 1996: {\color{RedOrange}RIPEMD-128} (Dobbertin \etal); broken (collisions) 2013 (Landelle and Peyrin)
\item 1996: {\color{Green}RIPEMD-160} (Dobbertin \etal); unbroken so far
\item 2001: {\color{Green}SHA-2} (NSA); unbroken so far
\end{itemize}
\end{itemize}
\end{frame}

\begin{frame}{Lesson to learn?}
\begin{itemize}
\item \textbf{DON'T USE BROKEN ALGORITHMS}
\item \textbf{MOVE AWAY FROM BROKEN ALGORITHMS}
\item \textbf{DO CARE ABOUT ``THEORETICAL'' ATTACKS}
\item \textbf{BROKEN CRYPTO IS NOT ``COOL''}
\end{itemize}

\medskip

Perfect bad example: Git
\begin{itemize}
\item Don't use SHA-1 in 2005!
\item Don't hide needed security properties!
\end{itemize}

Also:
\begin{itemize}
\item Don't use SHA-1, even if you just care about preimage attacks
\end{itemize}
\end{frame}

\begin{frame}{Back to design: how to do $\cf$?}
\begin{enumerate}
\item Start like a block cipher
\item Add feedforward to prevent invertibility
\end{enumerate}

\medskip

Examples:\\
``Davies-Meyer'': $\cf(h, m) = \E_m(h) \boxplus h$\\
``Matyas-Meyer-Oseas'': $\cf(h, m) = \E_h(m) \boxplus m$

\medskip

\begin{itemize}
\item Systematic analysis by Preneel, Govaerts and Vandewalle (1993). ``PGV'' constructions
\item Then rigorous proofs (in the ideal cipher model) (Black \etal, 2002), (Black \etal, 2010)
\end{itemize}
\end{frame}

\begin{frame}{Re: Davies-Meyer}
Picture:

\medskip

\begin{center}
\includegraphics[scale=1.5]{Davies-Meyer}
\end{center}

\medskip

Used in MD4/5 SHA-0/1/2, etc.
\end{frame}

\begin{frame}{Re: Re: Davies-Meyer}
\vspace{-4mm}

Why is the ``message'' the ``key''?
\begin{itemize}
\item Disconnect chaining value and message length!
\item $\E$'s block length: fixed by security level
\item $\E$'s key length: fixed by ``message'' length
\item Large ``key'' $\Rightarrow$ more efficient
\item Example: MD5's ``block cipher'' (also bad): 128-bit blocks, 512-bit keys
\end{itemize}

\medskip

DM incentive: use very simple \emph{message expansion} (``key schedules'')
\begin{itemize}
\item To be efficient!
\item Warning: can be a source of weakness (MD5, SHA-0, SHA-1. Should that be surprising?)
\end{itemize}
\end{frame}

\begin{frame}{Major PGV Warning}
PGV constructions are proved secure in the \emph{ideal cipher model}, \textbf{BUT}
\begin{itemize}
\item Real ciphers are not ideal!
\item Real ciphers \emph{don't have to be ideal} to be okay ciphers
\begin{itemize}
\item IDEA (Lai and Massey, 1991): weak key classes (Daemen \etal, 1993)
\item TEA (Wheeler and Needham, 1994): equivalent keys (Kelsey \etal, 1996)
\end{itemize}
\end{itemize}

\medskip

What can go wrong?
\end{frame}

\begin{frame}{Bad case of crypto design}
Microsoft needed a hash function for ROM integrity check of the XBOX
\begin{itemize}
\item Used TEA in DM mode (Steil, 2005)
\item {\footnotesize Because of an earlier break of their RC4-CBC-MAC scheme (ibid.)}
\item Terrible idea, because of existence of equivalent keys!
\item TEA$(k,m)$ = TEA$(\hat{k},m)$ $\Rightarrow$ DM-TEA$(h,k)$ = DM-TEA$(h,\hat{k})$ $\Rightarrow$ easy collisions!
\item Got hacked...
\item {\footnotesize IDEA for a hash function: also bad (Wei \etal, 2012)}
\end{itemize}

\medskip

\textbf{NEVER DESIGN YOUR OWN CRYPTO!}
\end{frame}

\begin{frame}{It's not all that bad, tho}
\begin{itemize}
\item AES in a PGV construction so far unbroken (see \eg Sasaki (2011))
\begin{itemize}
\item But small parameters \textinterrobang
\end{itemize}
\item Ditto, SHA-256 as a block cipher: ``SHACAL-2'' (Handschuh and Naccache, 2001)
\begin{itemize}
\item Enormous keys, 512 bit!
\end{itemize} 
\end{itemize}
\end{frame}

\begin{frame}{Now just a few examples}
\vspace{-6mm}

The MD4/MD5/SHA-0/SHA-1 family

\begin{itemize}
\item Merkle-Damg\aa rd construction 
\item Davies-Meyer compression function
\begin{itemize}
\item 512-bit messages, 128-bit (resp. 160-bit) chaining values for MD4/5 (resp. SHA-0/1)
\item Ad-hoc ``block cipher'' inside the compression function: \emph{unbalanced ARX Feistel}
\item Very simple message expansion
\end{itemize}
\item All broken (MD4 could be broken \emph{by hand})
\end{itemize}

RIPEMD family
\begin{itemize}
\item Somewhat similar, but uses two parallel lines merged at the end
\end{itemize}

SHA-2
\begin{itemize}
\item Somewhat similar, but heavier message expansion
\end{itemize}
\end{frame}

\begin{frame}{ARX? Whazzat?}

\vspace{-4mm}

ARX: Addition, Rotation, XOR
\begin{itemize}
\item A minimal set of operations to build (symmetric) cryptography
\item Used as early as 1987 in the FEAL block cipher (Shimizu and Miyaguchi)
\item Sometimes enriched with bitwise Boolean functions
\end{itemize}

\medskip

Example: SHA-1\\

\smallskip

$A_{i+1} = A_i^{\circlearrowleft 5} + \phi_{i\div20}(A_{i-1}, A_{i-2}^{\circlearrowright 2}, A_{i-3}^{\circlearrowright 2}) + A_{i-4}^{\circlearrowright 2} + W_i +K_{i\div20}$
\begin{itemize}
\item $A$: the internal state registers (32-bit)
\item $K$: round constant
\item $W$: expanded message word (the ``key'')
\item $\phi$: one of IF, MAJ, XOR
\end{itemize}
\end{frame}

\begin{frame}{SHA-1: alternate view}
\centering
\includegraphics[scale=0.9]{sha1}
\end{frame}

\begin{frame}{More on SHA: message expansion}
\vspace{-4mm}

SHA-1 has 80 steps, $W_i$s are 32-bit, we only have 512-bit messages\\
What are the 2560 bits of $W_{0,\ldots,79}$?
\begin{enumerate}
\item $W_{0,\ldots,15} = M_{0,\ldots,15}$ (512 bits of $M$) 
\item $W_i = (W_{i-3} \oplus W_{i-8} \oplus W_{i-14} \oplus W_{i-16})^{\circlearrowleft 1}$, $16 \leq i < 80$
\end{enumerate}

\smallskip

Note: SHA-0 is \textbf{exactly} like SHA-1, except for:
\begin{itemize}
\item $W_i = W_{i-3} \oplus W_{i-8} \oplus W_{i-14} \oplus W_{i-16}$, $16 \leq i < 80$
\end{itemize}

And yet...
\begin{itemize}
\item First (theoretical) collision attack:
\begin{itemize}
\item SHA-0 $\rightsquigarrow$ 1998
\item SHA-1 $\rightsquigarrow$ 2005
\end{itemize}
\item Fastest (actual) collision computation:
\begin{itemize}
\item SHA-0 $\rightsquigarrow$ 1 hour on a desktop PC (Manuel and Peyrin, 2008)
\item SHA-1 $\rightsquigarrow$ 110 year on a GPU (Stevens \etal, 2017)
\end{itemize}
\end{itemize}
\end{frame}

\begin{frame}{And now for something different}
If you need a hash function today $\Rightarrow$ SHA-3 (initially Keccak, (Bertoni \etal, 2008))
\begin{itemize}
\item Winner of an academic competition run by NIST (2008--2012)
\end{itemize}
\begin{itemize}
\item Sponge construction (not Merkle-Damgaard)
\item Based on a permutation (not a compression function)
\item Permutation is an SPN (not a Feistel, not ARX)
\end{itemize}

Sponge:
\begin{enumerate}
\item Compute $i \defas \perm(\perm(\ldots\perm(m_1||0^c)\oplus m_2||0^c)\ldots)$
\item Output $\hash(m) \defas \lfloor i \rfloor_r || \lfloor \perm(i) \rfloor_r || \ldots || \lfloor \perm^n(i) \rfloor_r$
\end{enumerate}

\end{frame}

\begin{frame}{Picture of a sponge}
\begin{center}
\includegraphics[scale=0.7]{sponge2}
\end{center}
\url{https://www.iacr.org/authors/tikz/}
\end{frame}

\begin{frame}{Sponge nice features}
\begin{itemize}
\item Indifferentiable from a RO (same, as Wide-pipe MD) (Bertoni \etal, 2008)
\item Quite flexible
\begin{itemize}
\item For fixed permutation size: speed/security tradeoff
\end{itemize}
\item Natively a XOF
\item Can be extended to do (authenticated) encryption
\item Simpler to design a permutation; less of a waste?
\end{itemize}

\begin{itemize}
\item Close structure: JH construction, another SHA-3 competitor (Wu, 2008)
\end{itemize}

\end{frame}

\begin{frame}{Conclusion}
\begin{itemize}
\item Don't design crypto yourself!
\begin{itemize}
\item There is no generic way to design a hash function
\item Every small detail counts (recall e.g. SHA-0, TEA)
\end{itemize}
\item Use SHA-3 (SHA-2 still okay)
\item Don't trust ISO standards
\item NEVER USE MD5/SHA-1
\begin{itemize}
\item Even if you only care about preimage attacks
\end{itemize}
\end{itemize}
\end{frame}

\end{document}
